\documentclass[8pt,a4paper]{moderncv}
\usepackage[utf8]{inputenc}
\usepackage[top=0.3in, left=0.3in, right=0.3in, bottom=0.8in]{geometry}
\moderncvtheme[black]{casual}
\usepackage{lmodern}
\usepackage[english,ngerman]{babel}
\usepackage{amsmath}
\usepackage[utf8]{inputenc}
\usepackage{relsize}
\usepackage{enumitem}
\setlength{\hintscolumnwidth}{4.25cm}

\def\locale{$LANG$}

\usepackage{ifthen}
\def\localedef#1#2{
  \ifthenelse{\equal{\locale}{#1}}{
    \selectlanguage{#2}
    \expandafter\def\csname#1\endcsname ##1{##1}
  }{
    \expandafter\def\csname#1\endcsname ##1{}
  }
}

\localedef{en}{english}
\localedef{de}{ngerman}

\firstname{Braden}
\familyname{Walters}
\title{\en{R\'{e}sum\'{e} / Curriculum Vitae}\ignorespaces
       \de{Resum\'e / Lebenslauf}}
\address{Reisewitzer Str. 39}{01159 Dresden, \en{Germany}\de{Deutschland}}
\mobile{+4915110464343}
\email{cv@braden-walters.info}
\photo[3cm]{images/DSCN1157}
\homepage{http://braden-walters.info}

\renewcommand*\namefont{\fontsize{40}{48}\selectfont}
\renewcommand*\titlefont{\fontsize{20}{24}\selectfont}
\renewcommand*\addressfont{\selectfont}
\renewcommand*\sectionfont{\fontsize{20}{24}\selectfont}
\renewcommand*\subsectionfont{\fontsize{14}{100}\selectfont}

\begin{document}
  \relscale{0.95}

  \maketitle

  \section{\textbf{\en{Work Experience}\de{Arbeitserfahrungen}}}
  \cventry{\en{Since Feb 2017}\de{Seit Feb 2017}}{HicknHack Software GmbH}
          {\en{Client projects in C++, C\#, and TypeScript with Qt, React, and
               Angular}\ignorespaces
           \de{Kundenprojekte in C++, C\# und TypeScript mit Qt, React und
               Angular}}{}{}{}
  \begin{itemize}[leftmargin=51mm, rightmargin=10mm, noitemsep, topsep=0pt]
    \item \en{Maintenance of graphical Qt/Windows application for controlling and programming theatre machinery.}
          \de{Wartung von graphischer Qt/Windows-Anwendung f\"ur Steuerung und Programmierung von Theater-Maschinen.}
    \item \en{Development of TypeScript/Rust application for reporting and analysing data from manufacturing systems.}
          \de{Entwicklung einer TypeScript/Rust-Anwendung f\"ur Berichtsgenerierung und Analyse von Daten aus Fertigungsanlagen.}
    \item \en{Development and maintenance of .NET/TypeScript application for controlling, programming, and reporting for automated manufacturing systems. Client runs in desktop and mobile environments.}
          \de{Entwicklung und Wartung einer .NET/TypeScript-Anwendung f\"ur Steuerung, Programmierung und Berichtsgenerierung f\"ur automatisierte Fertigungsanlagen. Client läuft in Desktop- und mobilen Umgebungen.}
  \end{itemize}
  \cventry{\en{Jun 2015 - Jan 2017}\de{Jun 2015 - Jan 2017}}
          {Unister Travel Betriebsgesellschaft mbH}
          {\en{Backend administration web software and \hbox{automation} for flight portals in PHP with Zend Framework}\ignorespaces
           \de{Weboberfl\"ache zur Backend-Verwaltung und Automation f\"ur Flugportale in PHP with Zend Framework}}{}{}{}
  \begin{itemize}[leftmargin=51mm, rightmargin=10mm, noitemsep, topsep=0pt]
    \item \en{Feature development of order fulfillment, customer support, and accounting systems.}
          \de{Feature-Entwicklung für Fulfillment-, Kundensupport- und Buchhaltungs-Systeme.}
    \item \en{Integration of new and existing computer reservation systems (CRS/GDS).}
          \de{Integration von neuen und schon vorhandenen Computerreservierungssystemen (CRS/GDS).}
    \item \en{Responded to business-critical outages and system errors to maintain system availability.}
          \de{Behebung gesch\"aftskritischer Ausf\"alle und Systemfehler zur Erhaltung der Systemverf\"ugbarkeit.}
    \item \en{Development of queries optimised for large data sets in MySQL to export and analyse data.}
          \de{Entwicklung von Abfragen in MySQL, die auf gro{\ss}e Datenmengen optimiert sind, um sie zu exportieren und analysieren.}
  \end{itemize}
  \cventry{\en{Sep 2013 - Feb 2015}\de{Sep 2013 - Feb 2015}}
          {Brilex Industries, Inc.}
          {\en{Web software for internal usage in C\# with Microsoft MVC}\ignorespaces
           \de{Websoftware f\"ur internen Gebrauch in C\# mit Microsoft MVC}}
          {}{}{}
  \begin{itemize}[leftmargin=51mm, rightmargin=10mm, noitemsep, topsep=0pt]
    \item \en{Development of web software using Microsoft MVC and NHibernate for database access.}
          \de{Entwicklung von Websoftware mit Microsoft MVC und NHibernate für Datenbankzugriff.}
    \item \en{Frontends for web applications developed using CoffeeScript and jQuery.}
          \de{Frontends für Webanwendungen mit CoffeesScript und jQuery entwickelt.}
  \end{itemize}
  \cventry{Nov 2011 - Sep 2014}{Learning Egg, LLC.}
          {\en{Development of the Lightning Grader in Yii/PHP and later Ruby on Rails}\ignorespaces
           \de{Entwicklung des Lightning Grader auf Yii/PHP und sp\"ater Ruby on Rails}}{}{}{}
  \begin{itemize}[leftmargin=51mm, rightmargin=10mm, noitemsep, topsep=0pt]
    \item \en{Development of features for customers focusing on both front-end and back-end.}
          \de{Entwicklung von Features f\"ur Kunden mit Fokus auf Front-End und Back-End.}
    \item \en{Maintenance and feature development in PHP with Yii.}
          \de{Wartung und Feature-Entwicklung in PHP mit Yii}
    \item \en{Optimisation and development of reporting queries in PostgreSQL.}
          \de{Optimierung und Entwicklung von Berichtsabfragen in PostgreSQL.}
    \item \en{Supported migration of application to Ruby on Rails.}
          \de{Unterst\"utzung bei Migration einer Anwendung zu Ruby on Rails geleistet.}
  \end{itemize}
  \cventry{\en{Oct 2011 - Mar 2013}\de{Okt 2011 - M\"ar 2013}}
          {\en{YSU Biology Department}\de{YSU Biologie-Fakult\"at}}
          {\en{Development of the Plant Alternative Splice Database in PHP}\ignorespaces
           \de{Entwicklung der Plant Alternative Splice Database in PHP}}{}{}{}
  \begin{itemize}[leftmargin=51mm, rightmargin=10mm, noitemsep, topsep=0pt]
    \item \en{Import of genome data acquired from external bioinformatic tools.}
          \de{Importfunktionen f\"ur Genomdaten aus externen Bioinformatikwerkzeugen.}
    \item \en{Development of minimal PHP application for searching and displaying data without a framework.}
          \de{Entwicklung einer minimalen PHP-Anwendung f\"ur Datensuche und Anzeige ohne Framework.}
  \end{itemize}

  \section{\textbf{\en{Education}\de{Ausbildung}}}
  \subsection{\en{Academic Education and Professional Development}\ignorespaces
              \de{Akademische Ausbildung und beruflicher Werdegang}}
  \cventry{\en{Aug 2011 - May 2015}\de{Aug 2011 - Mai 2015}}
          {Youngstown State University\de{ (Universit\"at)}}
          {Youngstown, Ohio}{}{}{}{}
  \begin{itemize}[leftmargin=51mm, noitemsep, topsep=0pt]
    \item \en{B.S Computer Science, Minor: Mathematics}\ignorespaces
          \de{B.Sc. Informatik, Nebenfach: Mathematik}
  \end{itemize}
  \cventry{}{\en{Grade Point Average}\de{Durchschnittsnote}}
            {\en{3.68 (US Grade Point Average)}\ignorespaces
             \de{3,68 (US-Grade-Point-Average), ca. 1,3 im dt. Notensystem}}
            {}{}{}

  \vspace{1mm}
  \subsection{\en{Language Skills}\de{Sprachkenntnisse}}
  \cvlanguage{\en{English}\de{Englisch}}
             {\en{Native Language}\de{Muttersprache}}{}
  \cvlanguage{\en{German}\de{Deutsch}}
             {\en{Advanced Knowledge (self-estimated C1)}\ignorespaces
              \de{Fortgeschrittene Kenntnis (C1 selbsteingesch\"atzt)}}{}
  \begin{itemize}[leftmargin=51mm, rightmargin=10mm, noitemsep, topsep=0pt]
    \item \en{TestDaF Exam Level 5 in Reading Comprehension, Listening Comprehension, and Verbal Expression}\ignorespaces
          \de{TestDaF Niveau 5 in Leseverstehen, H\"orverstehen und m\"undlichem Ausdruck}
    \item \en{TestDaF Exam Level 4 in Written Expression}\ignorespaces
          \de{TestDaF Niveau 4 in schriftlichem Ausdruck}
  \end{itemize}

  \pagebreak

  \section{\textbf{\en{Qualifications}\de{Qualifikationen}}}
  \subsection{\en{Software Development}\de{Softwareentwicklung}}
  \cventry{\en{Operating Systems}\de{Betriebssysteme}}{GNU/Linux}{}{}{}{}
  \cventry{\en{Programming Languages}\de{Programmiersprachen}}
          {C, C++, Python, Java, JavaScript, PHP, Ruby, CoffeeScript, C\#,
           \linebreak Clojure, Haskell, TypeScript}{}{}{}{}
  \cventry{\en{Software Concepts}\de{Softwarekonzepte}}
          {\en{Basic Game AI, Compiler Design}\ignorespaces
           \de{Grundlagen von Spiel-KI, Compilerbau}}{}{}{}{}
  \cventry{\en{Mathematics}\de{Mathematik}}
          {\en{Calculus, Matrix Theory, Group Theory}\ignorespaces
           \de{Integralrechnung, Matrixtheorie, Gruppentheorie}}{}{}{}{}
  \cventry{\en{General Tools}\de{Allgemeine Werkzeuge}}
          {Git, PostgreSQL, RabbitMQ, Reactive Extensions, EntityFramework 6}
          {}{}{}{}
  \cventry{\en{Server-Side Web Development}
           \de{Serverseitige Webentwicklung}}
          {Ruby on Rails, Django, Yii Framework, ASP.NET Core, Microsoft MVC, Zend Framework 1}
          {}{}{}{}
  \cventry{\en{Client-Side Web Development}
           \de{Clientseitige Webentwicklung}}
           {React, Flux, Angular, jQuery, Webpack}{}{}{}{}
  \cventry{\en{Desktop and Mobile Application Development}\ignorespaces
           \de{Desktop- und Mobile-Anwendungsentwicklung}}
          {\en{Qt, Basics of Android SDK for Media Applications}\ignorespaces
           \de{Qt, Grundlagen des Android-SDKs für Medienanwendungen}}{}{}{}{}
  \cventry{\en{Graphics Software / Game Development}
           \de{Grafiksoftware / Spielentwicklung}}
          {Unity3D, OpenGL 2.1}{}{}{}{}

  \section{\textbf{\en{Research Publications}\ignorespaces
                   \de{Publikationen in der Forschung}}}
  \subsection{\en{Computer Science}\de{Informatik}}
  \cventry{2015}
          {iTrace: Enabling Eye Tracking on Software Artifacts Within the IDE to Support Software Engineering Tasks}
          {Shaffer, T., Wise, J., Walters, B., M\"uller, S., Falcone, M., Sharif B.}
          {}{}{}
  \cventry{2015}
          {Tracing Software Developers’ Eyes and Interactions for Change Tasks}
          {Kevic, K., Walters, B., Shaffer, T., Sharif B., Fritz, T., Shepherd, D.}
          {}{}{}
  \cventry{2014}
          {Capturing Software Traceability Links from Developers Eye Gazes}
          {Walters, B., Shaffer, T., Sharif, B., and Kagdi, H.}
          {}{}{}
  \cventry{2013}
          {Towards an Eye-tracking Enabled IDE for Software Traceability Tasks}
          {Walters, B., Falcone, M., Shibble, A., Sharif. B.}
          {}{}{}
  \vspace{1mm}
  \subsection{\en{Bioinformatics}\de{Bioinformatik}}
  \cventry{2013}
          {Genome-wide landscape of alternative splicing events in Brachypodium distachyon}
          {Walters, B., Lum, G., Sablok, G., Min, XJ.}
          {}{}{}

  \section{\textbf{\en{Software Developed for Research}\ignorespaces
                   \de{Für Forschung entwickelte Software}}}
  \cventry{\en{Jan 2013 - Mar 2015}\de{Jan 2013 - M\"ar 2015}}
          {iTrace}
          {\en{Eclipse Plugin for Finding Software Traceability Links with Eye-Tracking Data}\ignorespaces
           \de{Eclipse-Plugin, das Software-Traceability-Links mithilfe von Eyetracking-Daten \hbox{ermittelt}}}
          {}{}{\url{http://www.i-trace.org/}}
  \cventry{\en{Oct 2011 - Mar 2013}\de{Okt 2011 - M\"ar 2013}}
          {Plant Alternative Splice Database}
          {\en{Publically Searchable Application to Search and Display Information about Alternative Splices in Plant Species}\ignorespaces
           \de{\"Offentlich zug\"angliche Anwendung zur Suche und Anzeige von Informationen \"uber alternative Splei{\ss}en in Pflanzenspezien}}
          {}{}{\url{http://proteomics.ysu.edu/altsplice/}}

  \section{\textbf{\en{Free and Open Source Software Projects}\ignorespaces
                   \de{Freie und Quelloffene Softwareprojekte}}}
  \cventry{\en{GitHub Profile}\de{GitHub-Profil}}
          {\url{https://github.com/meoblast001}}{}{}{}{}
  \cventry{\en{since Feb 2019}\de{seit Feb 2019}}
          {RocketThing}
          {\en{3D action game in Unity3D}\de{3D-Actionspiel in Unity3D}}{}{}{}
  \cventry{\en{since Nov 2013}\de{seit Nov 2013}}{Thugaim}
          {\en{2D video game for Android in Java}\ignorespaces
           \de{2D-Videospiel f\"ur Android in Java}}{}{}{}
  \cventry{\en{since Aug 2011}\de{seit Aug 2011}}{KKSystem}
          {\en{Digital flash card system originally on Django in Python and now in Ruby on Rails}\ignorespaces
           \de{Digitales Flashkartensystem urspr\"unglich auf Django in Python aber jetzt in Ruby on Rails}}{}{}{}

  \de{\pagebreak}

  \section{\textbf{\en{GameJam Projects}\de{GameJam-Projekte}}}
  \cventry{Nov 2019}
          {Busy Beaver (3m5 GameJam Nov 2019)}
          {\en{2D platformer game made in Godot. Contributions in light effects, art, and game logic}\ignorespaces
           \de{2D-Plattformspiel, das in Godot gemacht wurde. Zu Lichteffekten, Kunst und Spiellogik beigetragen}}{}{}
          {\url{https://github.com/meoblast001/3m5GameJamNovember2019}}
  \cventry{\en{May 2019}\de{Mai 2019}}
          {Island Escape (3m5 GameJam May 2019)}
          {\en{3D network multiplayer (2-player) action game made in Unity3D. Contributions in game logic and networking}\ignorespaces
           \de{3D-Netzwerkmultiplayer (2-Spieler) Actionspiel, das in Unity3D gemacht wurde. Zu Spiellogik und Networking beigetragen}}{}{}
          {\url{https://github.com/Rittergit/GameJamMay2019}}

\iffalse
  \section{\textbf{\en{Additional Information}\de{Weitere Informationen}}}
  \cventry{}{\en{This information and more is available at \url{http://braden-walters.info}}\ignorespaces
             \de{Diese und weitere Informationen sind unter \url{http://braden-walters.info} verf\"ugbar}}
          {}{}{}{}
\fi
\end{document}
