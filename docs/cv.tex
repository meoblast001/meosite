\documentclass[8pt,a4paper]{moderncv}
\usepackage[utf8]{inputenc}
\usepackage[top=0.5in, left=0.5in, right=0.5in, bottom=1.0in]{geometry}
% \usepackage[top=0.3in, left=0.3in, right=0.3in, bottom=0.8in]{geometry}
\moderncvtheme[black]{casual}
\usepackage{lmodern}
\usepackage[english,ngerman]{babel}
\usepackage{amsmath}
\usepackage[utf8]{inputenc}
\usepackage{relsize}
\usepackage{enumitem}
\setlength{\hintscolumnwidth}{4.25cm}

\usepackage{ifthen}
\def\localedef#1#2{
  \ifthenelse{\equal{\locale}{#1}}{
    \selectlanguage{#2}
    \expandafter\def\csname#1\endcsname ##1{##1}
  }{
    \expandafter\def\csname#1\endcsname ##1{}
  }
}

\localedef{en}{english}
\localedef{de}{ngerman}

\firstname{Braden}
\familyname{Walters}
\title{\en{R\'{e}sum\'{e} / Curriculum Vitae}\ignorespaces
       \de{Resum\'e / Lebenslauf}}
\address{Landsberger Str. 10}{01187 Dresden, \en{Germany}\de{Deutschland}}
\mobile{+4915110464343}
\email{cv@braden-walters.info}
\photo[3cm]{images/DSCN1157}
\homepage{http://braden-walters.info}

\renewcommand*\namefont{\fontsize{40}{48}\selectfont}
\renewcommand*\titlefont{\fontsize{20}{24}\selectfont}
\renewcommand*\addressfont{\selectfont}
\renewcommand*\sectionfont{\fontsize{20}{24}\selectfont}
\renewcommand*\subsectionfont{\fontsize{14}{100}\selectfont}

\begin{document}
  \relscale{0.95}

  \maketitle
  \section{\textbf{\en{Personal Information}\de{Pers\"onliche Angaben}}}
  \cventry{\en{Nationality}\de{Staatsangeh\"origkeit}}
          {\en{US American}\de{US-Amerikanisch}}{}{}{}{}
  \cventry{\en{Born}\de{Geburt}}
          {\en{23 February 1993}\de{23. Februar 1993}}
          {Youngstown, Ohio}{}{}{}
  \de{\cventry{Familienstand}{ledig}{}{}{}{}}

  \section{\textbf{\en{Education}\de{Ausbildung}}}
  \subsection{\en{Academic Education and Professional Development}\ignorespaces
              \de{Akademische Ausbildung und beruflicher Werdegang}}
  \cventry{\en{Aug 2011 - current}\de{Aug 2011 - heute}}
          {Youngstown State University\de{ (Universit\"at)}}
          {Youngstown, Ohio}{}{}{}{}
  \begin{itemize}[leftmargin=51mm, noitemsep, topsep=0pt]
    \item \en{B.S Computer Science}\de{B.Sc. Informatik}
    \item \en{Minor: Mathematics}\de{Nebenfach: Mathematik}
    \item \en{Graduation Date: May 2015}\ignorespaces
          \de{Abschlussdatum: Mai 2015}
  \end{itemize}
  \cventry{}{\en{GPA}\de{Durchschnittsnote}}
            {\en{3.68 on a 4.0 scale}\de{3,68 in der 4,0-Skala}}{}{}{}
  \cventry{Jun 2013 -- Aug 2013}{The Learning Egg, LLC}
          {\en{Internship as Software Developer using Ruby on
               Rails}\ignorespaces
           \de{Praktikum als Softwareentwickler mit Ruby on Rails}}{}{}{
      \begin{itemize}
        \item \en{Internship through NOCHE and PICAM}\ignorespaces
              \de{Praktikum durch NOCHE und PICAM}
      \end{itemize}
    }{}

  \vspace{1mm}
  \subsection{\en{Secondary School}\de{Schule}}
  \cventry{Aug 2007 - Jun 2011}{\en{High School}\de{Oberschule}}
          {McDonald High School}{McDonald, Ohio}{}{}

  \vspace{1mm}
  \subsection{\en{Language Skills}\de{Sprachkenntnisse}}
  \cvlanguage{\en{English}\de{Englisch}}
             {\en{Native Language}\de{Muttersprache}}{}
  \cvlanguage{\en{German}\de{Deutsch}}
             {\en{Advanced Knowledge (self-estimated C1)}\ignorespaces
              \de{Fortgeschrittene Kenntnis (C1 selbsteingesch\"atzt)}}
             {\en{Self-taught since Dec. 2010}\de{Selbstgelernt seit Dez. 2010}}
  \begin{itemize}[leftmargin=51mm, rightmargin=55mm, noitemsep, topsep=0pt]
    \item \en{TestDaF Exam Level 5 in Reading Comprehension, Listening
              Comprehension, and Verbal Expression}\ignorespaces
          \de{TestDaF Niveau 5 in Leseverstehen, H\"orverstehen und
              m\"undlichem Ausdruck}
    \item \en{TestDaF Exam Level 4 in Written Expression}\ignorespaces
          \de{TestDaF Niveau 4 in schriftlichem Ausdruck}
  \end{itemize}
  \cvlanguage{\en{Swedish}\de{Schwedisch}}
             {\en{Basic Knowledge (self-estimated A1)}\ignorespaces
              \de{Grundlegende Kenntnis (A1 selbsteingesch\"atzt)}}
             {\en{Self-taught since Aug. 2011}\de{Selbstgelernt seit Aug. 2011}}

  \section{\textbf{\en{Qualifications}\de{Qualifikationen}}}
  \subsection{\en{Software Development}\de{Softwareentwicklung}}
  \cventry{\en{Operating Systems}\de{Betriebssysteme}}{GNU/Linux}{}{}{}{}
  \cventry{\en{Programming Languages}\de{Programmiersprachen}}
          {C, C++, Python, Java, JavaScript, PHP, Ruby, CoffeeScript, C\#,
           \linebreak Clojure, Haskell}{}{}{}{}
  \cventry{\en{Software Concepts}\de{Softwarekonzepte}}
          {\en{Game AI, Compiler Design}\ignorespaces
           \de{Spiel-KI, Compilerbau}}{}{}{}{}
  \cventry{\en{Mathematics}\de{Mathematik}}
          {\en{Calculus, Matrix Theory, Group Theory}\ignorespaces
           \de{Integralrechnung, Matrixtheorie, Gruppentheorie}}{}{}{}{}
  \cventry{\en{General Tools}\de{Allgemeine Tools}}{Git, PostgreSQL}{}{}{}{}
  \cventry{\en{Tools for Web Development}\de{Tools f\"ur Webentwicklung}}
          {Ruby on Rails, Django, Yii Framework, Microsoft MVC, Zend Framework
           1, Hakyll, jQuery, Underscore.js}
          {}{}{}{}
  \cventry{\en{Tools for Mobile Development}\de{Tools f\"ur Mobileentwicklung}}
          {Android SDK/API}{}{}{}{}
  \cventry{\en{Tools for Graphics Software / Game Development}
           \de{Tools f\"ur Grafiksoftware / Spielentwicklung}}
          {OpenGL 2.1, OpenAL, Bullet Physics}{}{}{}{}

  \section{\textbf{\en{Work Experience}\de{Arbeitserfahrungen}}}
  \subsection{\en{Web Development}\de{Webentwicklung}}
  \cventry{\en{since Jun 2015}\de{seit Jun 2015}}{Unister Travel GmbH.}
          {\en{Backend administration web software and automation for flight
               portals in PHP}\ignorespaces
           \de{Backend-Verwaltungswebsofware und Automation f\"ur Flugportale
               auf PHP}}{}{}
          {}
  \cventry{\en{Sep 2013 - Feb 2015}\de{Sep 2013 - Feb 2015}}
          {Brilex Industries, Inc.}
          {\en{Web software for internal usage in C\# with Microsoft
               MVC}\ignorespaces
           \de{Websoftware f\"ur internen Gebrauch auf C\# mit Microsoft MVC}}
          {}{}{}
  \cventry{Nov 2011 - Sep 2014}{Learning Egg, LLC.}
          {\en{Development of the Lightning Grader in Yii/PHP and later Ruby on
               Rails}\ignorespaces
           \de{Entwicklung des Lightning Grader auf Yii/PHP und sp\"ater Ruby on
               Rails}}{}{}{}
  \cventry{\en{Oct 2011 - Mar 2013}\de{Okt 2011 - M\"ar 2013}}
          {\en{YSU Biology Department}\de{YSU Biologie-Abteilung}}
          {\en{Development of the Plant Alternative Splice Database in
               PHP}\ignorespaces
           \de{Entwicklung der Plant Alternative Splice Database auf PHP}}{}{}{}

  \section{\textbf{\en{Research Publications}\ignorespaces
                   \de{Publikationen in der Forschung}}}
  \subsection{\en{Computer Science}\de{Informatik}}
  \cventry{2015}{Shaffer, T., Wise, J., Walters, B., M\"uller, S., Falcone, M.,
                 Sharif B.}
          {iTrace: Enabling Eye Tracking on Software Artifacts Within the IDE to
           Support Software Engineering Tasks}
          {\en{10th Joint Meeting of the European Software Engineering
               Conference and the ACM SIGSOFT Symposium on the Foundations of
               Software Engineering (ESEC/FSE 2015), Tool Track, Bergamo, Italy,
               August 30 - September 4}
           \de{10. gemeinsames Treffen der europ\"aischen
               Softwaretecnik-Konferenz und die ACM SIGSOFT Fachtagung zu
               den Grundlagen der Softwaretechnik (ESEC/FSE 2015),
               Tool-Themenbereich Bergamo, Italien, 30. August - 4. September}}
          {}{}
  \cventry{2015}{Kevic, K., Walters, B., Shaffer, T., Sharif B., Fritz, T.,
                 Shepherd, D.}
          {Tracing Software Developers’ Eyes and Interactions for Change Tasks}
          {\en{10th Joint Meeting of the European Software Engineering
               Conference and the ACM SIGSOFT Symposium on the Foundations of
               Software Engineering (ESEC/FSE 2015), Bergamo, Italy, August 30 -
               September 4}
           \de{10. gemeinsames Treffen der europ\"aischen
               Softwaretecnik-Konferenz und die ACM SIGSOFT Fachtagung zu
               den Grundlagen der Softwaretechnik (ESEC/FSE 2015), Bergamo,
               Italien, 30. August - 4. September}}{}{}
  \cventry{2014}{Walters, B., Shaffer, T., Sharif, B., and Kagdi, H.}
          {Capturing Software Traceability Links from Developers Eye Gazes}
          {\en{International Conference on Program Comprehension (ICPC'14), ERA
               track, Hyderabad, India, June 2 – June 3 2014, pp.
               201-204.}\ignorespaces
           \de{Internationale Konferenz zu Programmverst\"andnis (ICPC’14),
               ERA-Themenbereich, Hyderabad, Indien, 2. Juni - 3. Juni 2014,
               S. 201-204}}{}{}
  \cventry{2013}{Walters, B., Falcone, M., Shibble, A., Sharif. B.}
          {Towards an Eye-tracking Enabled IDE for Software Traceability Tasks}
          {\en{7th International Workshop on Traceability in Emerging Forms of
               Software Engineering (TEFSE 2013). San Francisco, CA, 19 May
               2013. pp 51-54.}\ignorespaces
           \de{7ter internationaler Workshop zu Traceability in aufkommenden
               Formen der Softwaretechnik (TEFSE 2013). San Francisco, CA, 19
               Mai 2013. S. 51-54}}{}{}
  \vspace{1mm}
  \subsection{\en{Bioinformatics}\de{Bioinformatik}}
  \cventry{2013}{Walters, B., Lum, G., Sablok, G., Min, XJ.}
          {Genome-wide landscape of alternative splicing events in Brachypodium
           distachyon}
          {\en{DNA Res. Volume 20 Issue 2. pp 163-171.}\ignorespaces
           \de{DNS Forschung. Buch 20 Nummer 2. S. 163-171}}{}{}

  \section{\textbf{\en{Free and Open Source Software Projects}\ignorespaces
                   \de{Freie und Quelloffene Softwareprojekte}}}
  \cventry{\en{GitHub Profile}\de{GitHub-Profil}}
          {https://github.com/meoblast001}{}{}{}{}
  \cventry{\en{since Feb 2016}\de{seit Feb 2016}}
          {PayPal REST Client for Haskell}
          {\en{Library in Haskell which communicates with PayPal's REST
               API}\ignorespaces
           \de{Library in Haskell, die mit PayPals REST-API kommuniziert}}{}{}{}
  \cventry{\en{since Aug 2015}\de{seit Aug 2015}}{Hakyll Sass}
          {\en{Compiler binding for Hakyll which uses libsass to compile Sass
               files to CSS}\ignorespaces
           \de{\"Ubersetzeranbindung f\"ur Hakyll, der libsass verwendet, um
               Sass-Dateien zu CSS-Dateien zu \"ubersetzen}}{}{}{}
  \cventry{\en{since Nov 2013}\de{seit Nov 2013}}{Thugaim}
          {\en{2D video game for Android in Java}\ignorespaces
           \de{2D-Videospiel f\"ur Android in Java}}{}{}{}
  \cventry{\en{since Aug 2011}\de{seit Aug 2011}}{KKSystem}
          {\en{Digital flash card system originally on Django in Python and now
               in Ruby on Rails}\ignorespaces
           \de{Digitales Flashkartensystem urspr\"unglich auf Django in Python
               aber jetzt in Ruby on Rails}}{}{}{}

  \newpage

  \section{\textbf{\en{Achievements and Other Experience}\ignorespaces
                   \de{Leistungen und andere Erfahrungen}}}
  \subsection{\en{Study Abroad}\de{Auslandsstudium}}
  \cventry{Jan 2014 - Apr 2014}
          {\en{Studied in L\"uneburg, Germany through USAC}\ignorespaces
           \de{Studierte in L\"uneburg, Deutschland durch USAC}}
          {}{}{}{}
  \cventry{Nov 2013}{\en{Benjamin A. Gilman International
                         Scholarship}\ignorespaces
                     \de{Benjamin A. Gilman internationales Stipendium}}
          {\en{Awarded to study abroad in Spring 2014}\ignorespaces
           \de{Wurde f\"ur das Auslandsstudium in Fr\"uhling 2014 verliehen}}
          {}{}{}
  \vspace{1mm}
  \subsection{\en{Computer Science}\de{Informatik}}
  \cventry{\en{Dec 2012 - Dec 2014}\de{Dez 2012 - Dez 2014}}
          {\en{Eclipse Plugin iTrace}\de{Eclipse-Plugin iTrace}}
          {\en{Development of the plugin for traceability with eye trackers in
               the YSU SERESL lab}\ignorespaces
           \de{Entwicklung des Plugins f\"ur Traceability mit Eyetrackern im
               YSU-SERESL-Labor}}{}{}{}
  \cventry{\en{Jan 2012 - May 2013}\de{Jan 2012 - Mai 2013}}
          {\en{President of the Student Chapter of the ACM at YSU}\ignorespaces
           \de{Pr\"asident der Studentenortgruppe der ACM der YSU}}{}{}{}{}
  \cventry{Nov 2011, 2012, 2013}
          {\en{ACM International Collegiate Programming Competition
               (ICPC)}\ignorespaces
           \de{ACM Internationaler akedemischer Programmierungs-Wettbewerb
               (ICPC)}}
          {\en{Competed in competition}\de{Bei dem Wettbewerb angetreten}}
          {}{}{}
  \vspace{1mm}
  \subsection{\en{Academic Achievements}\de{Akedemische Leistungen}}
  \cventry{\en{Autumn}\de{Herbst} 2011, 2012,\\
           \en{Spring}\de{Fr\"uhling} 2013}
          {\en{Dean's List for Good Marks/Grades}\ignorespaces
           \de{In die Dean's List wegen guter Noten eingeschrieben}}{}{}{}{}

  \section{\textbf{\en{Presentations}\de{Pr\"asentationen}}}
  \cventry{\en{11 February 2015}\de{11. Februar 2015}}
          {An Introduction to Higher Order Functions}
          {YSU Association for Computing Machinery}{}{}{}
  \cventry{\en{3 November 2014}\de{3. November 2014}}
          {International Studies and Study Abroad}
          {\en{Presented four times in government and politics classes at
               McDonald High School}\ignorespaces
           \de{Viermal in Politikunterrichten an der Oberschule McDonald High
               School vorgetragen}}{}{}{}
  \cventry{\en{19 May 2013}\de{19. Mai 2013}}
          {Towards an Eye-Tracking Enabled IDE for Software Traceability Tasks}
          {TEFSE 2013}{}{}{}
\end{document}
